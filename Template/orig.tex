\documentclass[conference]{IEEEtran}
\IEEEoverridecommandlockouts
% The preceding line is only needed to identify funding in the first footnote. If that is unneeded, please comment it out.
\usepackage{cite}
\usepackage{amsmath,amssymb,amsfonts}
\usepackage{algorithmic}
\usepackage{graphicx}
\usepackage{textcomp}
\usepackage{xcolor}
\def\BibTeX{{\rm B\kern-.05em{\sc i\kern-.025em b}\kern-.08em
    T\kern-.1667em\lower.7ex\hbox{E}\kern-.125emX}}
\begin{document}

\title{Video Games Sales Analysis\\
}

\author{\IEEEauthorblockN{Dev Rajeshbhai Hathi}
    \IEEEauthorblockA{\textit{Department Of Computer Science And Engineering} \\
        \textit{PES University}\\
        Bangalore, India \\
        devrhathi@gmail.com}
    \and
    \IEEEauthorblockN{Harsha R Patil}
    \IEEEauthorblockA{\textit{Department Of Computer Science And Engineering} \\
        \textit{PES University}\\
        Bangalore, India \\
        email address or ORCID}
}


\maketitle

\begin{abstract}
    This project summerizes region specific impact of genre on the video games sales and the impact of exclusivity on gaming platform sales.  In the first stage of the project report, we aim to exhibit problem statement synopsis, explain the databases and relations among them by an Exploratory Data Analysis.\\
    In the latter stage, we further explore the dataset and get a thorough understanding, which will help us create a better model for prediction of the sales.
\end{abstract}

\begin{IEEEkeywords}
    video games, sales, consoles, exclusivity, genre, region, sales prediction
\end{IEEEkeywords}

\section{Introduction}
The video game industry has grown from niches to mainstream in past few years, having a significant contribution on the global sales market. The number of video games being developed, is showing an exponential growth. Data analysis for video game business can become large saled, personalized and real-time. The
statistical processing and Exploratory Data Analysis (EDA) are mainly performed on the data of game platforms, years of distribution and genres.\\
There are number of factors that can potentially affect the sales. Some of them would be:
\begin{itemize}
    \item Popularity of respective consoles
    \item Region
    \item Marketing budget
    \item Genre
    \item Number of positive reviews
    \item Required technical specifications
    \item Popularity of respective platform
\end{itemize}
In our project, we will be estimating the sales of video games and the sales of consoles (platforms). The analysis of video games sales, we will consider Region and Genre and for the sales of respective platforms, we will conduct a special analysis for which, exclusive games will be a primary factor for the sales.\\ The reason for considering exclusive games is because a very popular game being exclusive to a certain platform can potentially drive a lot of console sales since the primary use of a console is to play video games.\\
In the latter stage, we will infere the resultant statistics from the predictions made and make various conclusions such as prioritizing genres or regions for high sales. Which could help optimize marketing strategy and can potentially decrease marketing budget.\\
This project demonstrates that it is a worthwhile direction to analyse the sales for the investigation of business strategies for the publishers in conjunction with the market.



\section{Literature Review}
Before you begin to format your paper, first write and save the content as a
separate text file. Complete all content and organizational editing before
formatting. Please note sections \ref{AA}--\ref{SCM} below for more information on
proofreading, spelling and grammar.

Keep your text and graphic files separate until after the text has been
formatted and styled. Do not number text heads---{\LaTeX} will do that
for you.

\subsection{Abbreviations and Acronyms}\label{AA}
Define abbreviations and acronyms the first time they are used in the text,
even after they have been defined in the abstract. Abbreviations such as
IEEE, SI, MKS, CGS, ac, dc, and rms do not have to be defined. Do not use
abbreviations in the title or heads unless they are unavoidable.

\subsection{Units}
\begin{itemize}
    \item Use either SI (MKS) or CGS as primary units. (SI units are encouraged.) English units may be used as secondary units (in parentheses). An exception would be the use of English units as identifiers in trade, such as ``3.5-inch disk drive''.
    \item Avoid combining SI and CGS units, such as current in amperes and magnetic field in oersteds. This often leads to confusion because equations do not balance dimensionally. If you must use mixed units, clearly state the units for each quantity that you use in an equation.
    \item Do not mix complete spellings and abbreviations of units: ``Wb/m\textsuperscript{2}'' or ``webers per square meter'', not ``webers/m\textsuperscript{2}''. Spell out units when they appear in text: ``. . . a few henries'', not ``. . . a few H''.
    \item Use a zero before decimal points: ``0.25'', not ``.25''. Use ``cm\textsuperscript{3}'', not ``cc''.)
\end{itemize}

\subsection{Figures and Tables}
\paragraph{Positioning Figures and Tables} Place figures and tables at the top and
bottom of columns. Avoid placing them in the middle of columns. Large
figures and tables may span across both columns. Figure captions should be
below the figures; table heads should appear above the tables. Insert
figures and tables after they are cited in the text. Use the abbreviation
``Fig.~\ref{fig}'', even at the beginning of a sentence.

\begin{table}[htbp]
    \caption{Table Type Styles}
    \begin{center}
        \begin{tabular}{|c|c|c|c|}
            \hline
            \textbf{Table} & \multicolumn{3}{|c|}{\textbf{Table Column Head}}                                                         \\
            \cline{2-4}
            \textbf{Head}  & \textbf{\textit{Table column subhead}}           & \textbf{\textit{Subhead}} & \textbf{\textit{Subhead}} \\
            \hline
            copy           & More table copy$^{\mathrm{a}}$                   &                           &                           \\
            \hline
            \multicolumn{4}{l}{$^{\mathrm{a}}$Sample of a Table footnote.}
        \end{tabular}
        \label{tab1}
    \end{center}
\end{table}

\begin{figure}[htbp]
    \caption{Example of a figure caption.}
    \label{fig}
\end{figure}

Figure Labels: Use 8 point Times New Roman for Figure labels. Use words
rather than symbols or abbreviations when writing Figure axis labels to
avoid confusing the reader. As an example, write the quantity
``Magnetization'', or ``Magnetization, M'', not just ``M''. If including
units in the label, present them within parentheses. Do not label axes only
with units. In the example, write ``Magnetization (A/m)'' or ``Magnetization
\{A[m(1)]\}'', not just ``A/m''. Do not label axes with a ratio of
quantities and units. For example, write ``Temperature (K)'', not
``Temperature/K''.

\section{Dataset}
The datasets used in this project were scraped from ``vgchartz.com''.
\begin{itemize}
    \item vgsales - Video Games Sales
    \item console - Console Sales
\end{itemize}
\begin{table}[htbp]
    \caption{Video Games Sales}
    \begin{tabular}{ll}
        \hline
        \multicolumn{1}{c}{Field Name} & Description                                       \\ \hline
        Rank                           & Ranking of overall sales                          \\
        Name                           & The games name                                    \\
        Platform                       & Platform of the games release (i.e. PC,PS4, etc.) \\
        Year                           & Year of the game's release                        \\
        Genre                          & Genre of the game                                 \\
        Publisher                      & Publisher of the game                             \\
        NA\_Sales                      & Sales in North America (in millions)              \\
        EU\_Sales                      & Sales in Europe (in millions)                     \\
        JP\_Sales                      & Sales in Japan (in millions)                      \\
        Other\_Sales                   & Sales in the rest of the world (in millions)      \\
        Global\_Sales                  & Total worldwide sales.                            \\ \hline
    \end{tabular}
\end{table}

\begin{table}[htbp]
    \caption{Console Sales}
    \begin{tabular}{ll}
        \hline
        \multicolumn{1}{c}{Field Name} & Description                           \\ \hline
        ConsoleID                      & Console Name corresponding to vgsales \\
        Console\_Name                  & Actual console names                  \\
        Manufacturer                   & Console manufacturer                  \\
        Release\_Year                  & Console Release Year                  \\
        Sales                          & Number of units sold                  \\
        Type                           & Type of console: home or handheld
    \end{tabular}
\end{table}
Here, the ConsoleID is used as a foreign key for joining the video games sales dataset to this dataset and the number of sales are aquired from wikipedia.

\section*{Acknowledgment}
We would like to convey our gratitude to Dr. Nage Gowda for his support and assistance in the completion of this project. We would also like to thank the Computer Science and Engineering Department of PES University for encouraging us with this wonderful opportunity to work on a real world data analysis project with the. We are also thankful to the Teaching Assistants for their involvement and contribution in this course and help us interactively learn new concepts.

\begin{thebibliography}{00}
    \bibitem{b1} Statistical yearbook: cinema, television, video, and new media in Europe, Volume 1999. Council of Europe. 1996. p. 123. ISBN 9789287129048.
\end{thebibliography}
\end{document}
